\section*{Реализация на C++}
\addcontentsline{toc}{section}{Реализация на C++}
Динамическая библиотека (DLL) представляет собой библиотеку функций (ресурсов), которыми может пользоваться любой процесс, загрузивший эту библиотеку.\\
Динамическая библиотека libPendulum.dll содержит в себе классы моделей маятников, а так же некоторые необходимые константы, функции и т.д. Все физические величины далее приведены в СИ, если не указано другое. Пока что здесь всего три более-менне полноценных модели: простой математический маятник, физический маятник, и маятник, на который действует вязкое трение. Кроме того, метод для нахождения координат маятника Фуко. \\
\subsection*{Структура}
\addcontentsline{toc}{subsection}{Структура}
{\centering {\bfseries Составляющие} (на данный момент, еще можно добавить пользовательские модули):}
\begin{center}
\begin{tikzpicture}
\path (0, 0) node(x) {Pendulum.h}
(0,-2) node(p) {Пользовательский модуль (?)}
(4.5,-2) node(y) {Pendulum.cpp}
(-7.5, -2) node(z) {SMP.cpp}
(-4.5,-2) node(w) {PHYP.cpp }
(7.5,-2) node(u) {PWF.cpp};
\draw [->, mycol1, thick] (x) -- (p);
\draw [->, mycol1, thick] (x) -- (y);
\draw [->, mycol1, thick] (x) -- (z);
\draw [->, mycol1, thick] (x) -- (w);
\draw [->, mycol1, thick] (x) -- (u);
\end{tikzpicture}
\end{center}
{\centering \bfseries Внутренние зависимости:}
\begin{center}
\begin{tikzpicture}
\path (0, 0) node(x)  {Pendulum.cpp <=> базовый класс Pendulum}
(-5, -1) node(z) {SMP.cpp <=> класс Simple\_Math\_Pendulum}
(-5,-2) node(w) {PHYP.cpp <=> класс Ph\_Pendulum}
(5,-1) node(u) {PWF.cpp <=> класс Pendulum\_W\_Friction};
\draw [->, mycol1, thick] (x) -- (z);
\draw [->, mycol1, thick] (x) -- (u);
\draw [->, mycol1, thick] (z) -- (w);
\end{tikzpicture}
\end{center}
Чтобы подключить библиотеку, достаточно заголовочного файла Pendulum.h и, собственно, библиотеки.\\
Подробное описание кода см. документацию по проекту.
\subsection*{Пример работы}
\addcontentsline{toc}{subsection}{Пример работы (для Comlex\_Test.cpp}
\begin{figure}
\begin{minipage}[h!]{0.49\linewidth}
\texttt{
input.txt:\\
10 0.471239\\
1.5 0.1 15 2\\
1.5 0.1 15 2 0.0000186\\
1.5 0.1 15 2 куб 1\\
}
\end{minipage}
\begin{minipage}[h!]{0.49\linewidth}
\texttt{
output.txt:\\
Координата простого математического маятника:\\
2.06304\\
Координата маятника с вязким трением:\\
2.06129\\
Координата физического маятника:\\
2.60926\\
Координаты маятника Фуко:\\
x = 2.06304; y = 0.000682892\\
}
\end{minipage}
\end{figure}
На основании полученных данных можно сказать, что для маленького промежутка времени $t$ координаты представленных маятников не сильно отличаются друг от друга.