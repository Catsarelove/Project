\section*{Введение}
\addcontentsline{toc}{section}{Введение}
\parМатематическая модель является идеализированным и упрощенным ``заместителем'' реального объекта или явления, отражающим в математической форме важнейшие свойства оригинала --- законы, которым он подчиняется, связи, присущие его частям. Исследуя эти свойства, можно получить некоторую информацию об исходном объекте. Цель этого проекта --- моделирование траектории маятника. 
\parМаятник --- это система, подвешенная в поле тяжести и совершающая механические колебания. В этой работе я моделировала простейший маятник --- тело, подвешенное на нерастяжимой нити. Я рассмотрела разные варианты движения, в зависимости от сил, действие которых на маятник я учитывала.
\parТраектория маятника в данном случае --- это уравнение движения, которое определяет координату x (отклонение от вертикали) в зависимости от времени. В случае с маятником Фуко, плоскость колебаний которого медленно поворачивается, появляется так же координата y (траектория задается параметрически).
\parВсе представленные модели линейные (уравнения движения являются комплексными экспонентами (в т.ч. синусоидами)) и детерминированные (для заданных входных данных результат уникальный и предопределенный). \\
Модели маятников были написаны на C++ и собраны в динамическую библиотеку \textbf{libPendulum.dll}
\subsection*{Цель и задачи}
\addcontentsline{toc}{subsection}{Введение}
\begin{itemize}
		\item Цель работы --- моделирование траектории маятника.
		\item Задачи:
			\begin{itemize}
					\item Рассмотреть разные варианты движения маятника, в зависимости от действующих на него сил
					\item Реализовать каждую модель на C++, опираясь на полученные (найденные) теоретические результаты
					\item Собрать все в библиотеку для дальнейшего использования
			\end{itemize}
\end{itemize}

